% Prämabel
\documentclass[11pt,a4paper]{article}

% Codage
\usepackage[utf8]{inputenc}
\usepackage[T1]{fontenc}
\usepackage{siunitx}
\usepackage{indentfirst}

% Langue
\usepackage[english]{babel}

% Supplément
\usepackage{amsmath,amsfonts,amssymb}
\usepackage{verbatim} % pour faire des commentaires avec \begin{comment}...
\usepackage{float} % pour positioner un image exacetement où on veut
\usepackage
  [separate-uncertainty = true,
  multi-part-units = repeat]
  {siunitx} % Exemple \SI{0}{\kg \cdot \m^{-3}}

% Images
\usepackage[pdftex]{graphicx}
\usepackage{graphics}
\usepackage{subcaption} % pour positioner des figures côte à côte
\usepackage{wrapfig}

% pour l'inclusion de liens dans le document 
\usepackage[colorlinks,bookmarks=false,linkcolor=blue,urlcolor=blue]{hyperref}


% la mise en page
\usepackage{geometry}
\paperheight=297mm
\paperwidth=210mm

\pagestyle{plain}



% nouvelles commandes LaTeX, utilis\'ees comme abreviations utiles
\newcommand{\mail}[1]{{\href{mailto:#1}{#1}}}
\newcommand{\ftplink}[1]{{\href{ftp://#1}{#1}}}

%%%%%%%%%%%%%%%%%%%%%%%%%%%%%%%%%%%%%%%%%%%%%%%%%%%%%%%%
\begin{document}


% Le titre, l'auteur et la date
\title{Numerical Exercise \#4}
\author{XXX YYYY\\  % \\ pour fin de ligne
}
\date{\today}
\maketitle
%%%%%%%%%%%%%%%%%%%%%%%%%%%%%%%%%%%%%%%%%%%%%%%%%%%%%%%%



\newpage
\section{Question \#1 Initial Loading Pattern}

keff (scientific format with 5 significant digits): \\

peak power (scientific format with 5 significant digits): \\

fast flux at the core boundary (scientific format with 5 significant digits): \\

\begin{figure}[h]
	%\includegraphics[width=7cm]{IMage/SSjeff33_Al_geom1.png}
	\centering
	\caption{Normalized fast and thermal fluxes}
\end{figure}

\begin{figure}[h]
	%\includegraphics[width=7cm]{IMage/SSjeff33_Al_geom1.png}
	\centering
	\caption{Relative power distribution}
\end{figure}

Explanation of the main features of the power distribution:\\


\section{Question \#2 Optimized Loading Pattern}

\begin{table}[H]
	\centering
	\begin{tabular}{|c|c|c|c|c|c|c|}
		\hline
		1& 1& 1& 1& 1& 1& 4\\
		\hline
	\end{tabular}
	\caption{Loading pattern from the core center (left to right)}
\end{table}

keff (scientific format with 5 significant digits): \\

peak power (scientific format with 2 significant digits): \\

fast flux at the core boundary (scientific format with 2 significant digits): \\

\begin{figure}[h]
	%\includegraphics[width=7cm]{IMage/SSjeff33_Al_geom1.png}
	\centering
	\caption{Normalized fast and thermal fluxes}
\end{figure}

\begin{figure}[h]
	%\includegraphics[width=7cm]{IMage/SSjeff33_Al_geom1.png}
	\centering
	\caption{Relative power distribution}
\end{figure}



%%%%%%%






\end{document}
