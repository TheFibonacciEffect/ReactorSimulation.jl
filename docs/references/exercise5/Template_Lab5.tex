% Prämabel
\documentclass[11pt,a4paper]{article}

% Codage
\usepackage[utf8]{inputenc}
\usepackage[T1]{fontenc}
\usepackage{siunitx}
\usepackage{indentfirst}

% Langue
\usepackage[english]{babel}

% Supplément
\usepackage{amsmath,amsfonts,amssymb}
\usepackage{verbatim} % pour faire des commentaires avec \begin{comment}...
\usepackage{float} % pour positioner un image exacetement où on veut
\usepackage
  [separate-uncertainty = true,
  multi-part-units = repeat]
  {siunitx} % Exemple \SI{0}{\kg \cdot \m^{-3}}

% Images
\usepackage[pdftex]{graphicx}
\usepackage{graphics}
\usepackage{subcaption} % pour positioner des figures côte à côte
\usepackage{wrapfig}

% pour l'inclusion de liens dans le document 
\usepackage[colorlinks,bookmarks=false,linkcolor=blue,urlcolor=blue]{hyperref}


% la mise en page
\usepackage{geometry}
\paperheight=297mm
\paperwidth=210mm

\pagestyle{plain}



% nouvelles commandes LaTeX, utilis\'ees comme abreviations utiles
\newcommand{\mail}[1]{{\href{mailto:#1}{#1}}}
\newcommand{\ftplink}[1]{{\href{ftp://#1}{#1}}}

%%%%%%%%%%%%%%%%%%%%%%%%%%%%%%%%%%%%%%%%%%%%%%%%%%%%%%%%
\begin{document}


% Le titre, l'auteur et la date
\title{Numerical Exercise \#5}
\author{XXX YYYY\\  % \\ pour fin de ligne
}
\date{\today}
\maketitle
%%%%%%%%%%%%%%%%%%%%%%%%%%%%%%%%%%%%%%%%%%%%%%%%%%%%%%%%



\newpage
\section{Question \#1 Analytical solution}


\begin{table}[H]
	\centering
	\begin{tabular}{|l|c|}
		\hline
		U235& \\
		U238& \\
		Pu239& \\
		X& \\
		Y& \\
		\hline
	\end{tabular}
	\caption{Initial atomic concentrations in $at/cm^{-3}$}
\end{table}

\section*{System of Equations}

The system of equations to solve is:

\[
\begin{aligned}
1. \quad \frac{dC_5}{dt} &= \Phi \left(-\sigma_c^{\text{U-235}} - \sigma_f^{\text{U-235}}\right) C_5, \\
2. \quad \frac{dC_8}{dt} &= \Phi \left(-\sigma_c^{\text{U-238}} - \sigma_f^{\text{U-238}}\right) C_8, \\
3. \quad \frac{dC_9}{dt} &= \Phi \left(-\sigma_c^{\text{Pu-239}} - \sigma_f^{\text{Pu-239}}\right) C_9 + \Phi \sigma_c^{\text{U-238}} C_8, \\
4. \quad \frac{dC_X}{dt} &= -\Phi \sigma_c^X C_X + \Phi \sigma_f^{\text{U-235}} C_5 \eta^X - \lambda^X C_X, \\
5. \quad \frac{dC_Y}{dt} &= -\Phi \sigma_c^Y C_Y + \Phi \sigma_f^{\text{U-235}} C_5 \eta^Y - \lambda^Y C_Y.
\end{aligned}
\]


\subsection*{\(C_5(t)\):}

The equation for \(C_5(t)\) is:

\[
\frac{dC_5}{dt} = \Phi \left(-\sigma_c^{\text{U-235}} - \sigma_f^{\text{U-235}}\right) C_5.
\]

The solution is:

\[
C_5(t) = C_5(0) e^{\Phi \left(-\sigma_c^{\text{U-235}} - \sigma_f^{\text{U-235}}\right) t}.
\]

\subsection*{\(C_8(t)\):}

The equation for \(C_8(t)\) is:

\[
\frac{dC_8}{dt} = \Phi \left(-\sigma_c^{\text{U-238}} - \sigma_f^{\text{U-238}}\right) C_8.
\]

The solution is:

\[
C_8(t) = C_8(0) e^{\Phi \left(-\sigma_c^{\text{U-238}} - \sigma_f^{\text{U-238}}\right) t}.
\]

The equation for \(C_9(t)\) is:

\[
\frac{dC_9}{dt} = \Phi \left(-\sigma_c^{\text{Pu-239}} - \sigma_f^{\text{Pu-239}}\right) C_9 + \Phi \sigma_c^{\text{U-238}} C_8.
\]

Substitute the solution for \(C_8(t)\):

\[
C_8(t) = C_8(0) e^{\Phi \left(-\sigma_c^{\text{U-238}} - \sigma_f^{\text{U-238}}\right) t}.
\]

The equation becomes:

\[
\frac{dC_9}{dt} = \Phi \left(-\sigma_c^{\text{Pu-239}} - \sigma_f^{\text{Pu-239}}\right) C_9 + \Phi \sigma_c^{\text{U-238}} C_8(0) e^{\Phi \left(-\sigma_c^{\text{U-238}} - \sigma_f^{\text{U-238}}\right) t}.
\]

Let:

\[
\kappa_{\text{Pu-239}} = \Phi \left(\sigma_c^{\text{Pu-239}} + \sigma_f^{\text{Pu-239}}\right), \quad
\kappa_{\text{U-238}} = \Phi \left(\sigma_c^{\text{U-238}} + \sigma_f^{\text{U-238}}\right).
\]

The solution is:

\[
C_9(t) = C_9(0) e^{-\kappa_{\text{Pu-239}} t} + \frac{\Phi \sigma_c^{\text{U-238}} C_8(0)}{\kappa_{\text{Pu-239}} - \kappa_{\text{U-238}}} \left(e^{-\kappa_{\text{U-238}} t} - e^{-\kappa_{\text{Pu-239}} t}\right).
\]

\subsection*{\(C_X(t)\):}

The equation for \(C_X(t)\) is:

\[
\frac{dC_X}{dt} = -\Phi \sigma_c^X C_X - \lambda^X C_X + \Phi \sigma_f^{\text{U-235}} C_5 \eta^X.
\]

Combine terms:

\[
\frac{dC_X}{dt} = -\left(\Phi \sigma_c^X + \lambda^X\right) C_X + \Phi \sigma_f^{\text{U-235}} \eta^X C_5.
\]

Let:

\[
\kappa_X = \Phi \sigma_c^X + \lambda^X.
\]

Substitute \(C_5(t)\):

\[
C_5(t) = C_5(0) e^{\Phi \left(-\sigma_c^{\text{U-235}} - \sigma_f^{\text{U-235}}\right) t}.
\]

The equation becomes:

\[
\frac{dC_X}{dt} = -\kappa_X C_X + \Phi \sigma_f^{\text{U-235}} \eta^X C_5(0) e^{\Phi \left(-\sigma_c^{\text{U-235}} - \sigma_f^{\text{U-235}}\right) t}.
\]

The solution is:

\[
C_X(t) = C_X(0) e^{-\kappa_X t} + \frac{\Phi \sigma_f^{\text{U-235}} \eta^X C_5(0)}{\kappa_X - \Phi \left(\sigma_c^{\text{U-235}} + \sigma_f^{\text{U-235}}\right)} \left(e^{\Phi \left(-\sigma_c^{\text{U-235}} - \sigma_f^{\text{U-235}}\right) t} - e^{-\kappa_X t}\right).
\]

\subsection*{\(C_Y(t)\):}

The equation for \(C_Y(t)\) is:

\[
\frac{dC_Y}{dt} = -\Phi \sigma_c^Y C_Y - \lambda^Y C_Y + \Phi \sigma_f^{\text{U-235}} C_5 \eta^Y.
\]

This is identical in form to the equation for \(C_X(t)\). The solution is:

\[
C_Y(t) = C_Y(0) e^{-\kappa_Y t} + \frac{\Phi \sigma_f^{\text{U-235}} \eta^Y C_5(0)}{\kappa_Y - \Phi \left(\sigma_c^{\text{U-235}} + \sigma_f^{\text{U-235}}\right)} \left(e^{\Phi \left(-\sigma_c^{\text{U-235}} - \sigma_f^{\text{U-235}}\right) t} - e^{-\kappa_Y t}\right),
\]

where:

\[
\kappa_Y = \Phi \sigma_c^Y + \lambda^Y.
\]

The solutions to the system are:

\[
\begin{aligned}
C_5(t) &= C_5(0) e^{\Phi \left(-\sigma_c^{\text{U-235}} - \sigma_f^{\text{U-235}}\right) t}, \\
C_8(t) &= C_8(0) e^{\Phi \left(-\sigma_c^{\text{U-238}} - \sigma_f^{\text{U-238}}\right) t}, \\
C_9(t) &= C_9(0) e^{-\kappa_{\text{Pu-239}} t} + \frac{\Phi \sigma_c^{\text{U-238}} C_8(0)}{\kappa_{\text{Pu-239}} - \kappa_{\text{U-238}}} \left(e^{-\kappa_{\text{U-238}} t} - e^{-\kappa_{\text{Pu-239}} t}\right), \\
C_X(t) &= C_X(0) e^{-\kappa_X t} + \frac{\Phi \sigma_f^{\text{U-235}} \eta^X C_5(0)}{\kappa_X - \Phi \left(\sigma_c^{\text{U-235}} + \sigma_f^{\text{U-235}}\right)} \left(e^{\Phi \left(-\sigma_c^{\text{U-235}} - \sigma_f^{\text{U-235}}\right) t} - e^{-\kappa_X t}\right), \\
C_Y(t) &= C_Y(0) e^{-\kappa_Y t} + \frac{\Phi \sigma_f^{\text{U-235}} \eta^Y C_5(0)}{\kappa_Y - \Phi \left(\sigma_c^{\text{U-235}} + \sigma_f^{\text{U-235}}\right)} \left(e^{\Phi \left(-\sigma_c^{\text{U-235}} - \sigma_f^{\text{U-235}}\right) t} - e^{-\kappa_Y t}\right).
\end{aligned}
\]

Governing differential equations for each isotope (U235,U238,Pu239, X and Y):
\begin{equation}
	\frac{dN_{U235}}{dt} = ...; 
	\frac{dN_{U238}}{dt} = ...; 
	\frac{dN_{Pu239}}{dt} = ...; 	
	\frac{dN_{X}}{dt} = ...; 
	\frac{dN_{Y}}{dt} = ...;	 
\end{equation}
\begin{figure}[h]
	%\includegraphics[width=7cm]{IMage/SSjeff33_Al_geom1.png}
	\centering
	\caption{Evolution of the various isotopes concentrations in a semilog-Y scale between 0 and 365 days}
\end{figure}
\begin{figure}[h]
	%\includegraphics[width=7cm]{IMage/SSjeff33_Al_geom1.png}
	\centering
	\caption{Evolution of the various isotopes concentrations in a semilog-Y scale between 0 and 1 day}
\end{figure}

\section{Question \#2 Numerical solution}
\begin{figure}[h]
	%\includegraphics[width=7cm]{IMage/SSjeff33_Al_geom1.png}
	\centering
	\caption{Evolution of the various isotopes concentrations in a semilog-Y scale  between 0 and 365 days with the Forward Euler method and a time-step of 1h}
\end{figure}

\begin{figure}[h]
	%\includegraphics[width=7cm]{IMage/SSjeff33_Al_geom1.png}
	\centering
	\caption{convergence rate of Euler and Matrix exponential methodsin terms of time steps, varying the time step between 1min and 6h. The quantity of interest is the concentration of X after 1 day.}
\end{figure}


\section{Question \#3 Advanced Problem}
 
\begin{figure}[h]
	%\includegraphics[width=7cm]{IMage/SSjeff33_Al_geom1.png}
	\centering
	\caption{Evolution of the flux level ($\phi [cm^{-2}.s^{-1}]$) as a function of time.}
	\label{err}
\end{figure}
Explain the observed trends in Figure \ref{err} 

\begin{figure}[h]
	%\includegraphics[width=7cm]{IMage/SSjeff33_Al_geom1.png}
	\centering
	\caption{Evolution of the various isotopes concentrations in a semilog-Y scale  between 0 and 365 days.}
	\label{err2}
\end{figure}



%%%%%%%






\end{document}
